\documentclass[12pt, twoside, a4paper]{article}
\usepackage[T2A]{fontenc}
\usepackage[russian]{babel}
\usepackage[utf8]{inputenc}
\usepackage[a4paper,width=175mm,top=25mm,bottom=25mm,headsep=10pt]{geometry}
\usepackage{amsmath, amssymb, amsthm}
\usepackage{derivative}
\usepackage{libertine}
\usepackage{euler}
\usepackage[T1]{fontenc}
\setlength{\parindent}{1em}
\usepackage{fancyhdr}
\pagestyle{fancy}
\headheight=15pt
\fancyhf{}
\renewcommand{\headrulewidth}{0pt}
\fancyhead[CE]{\small ВВЕДЕНИЕ В ТЕОРИЮ НЕЛИНЕЙНЫХ ДИССИПАТИВНЫХ СИСТЕМ}
\fancyhead[CO]{\small Д.С. ЗАГОРУЛЯ}
\fancyhead[LE, RO]{\thepage}
\thispagestyle{fancy}
%\renewcommand{\thesection}{}
\title{\large \textbf{ВВЕДЕНИЕ В ТЕОРИЮ НЕЛИНЕЙНЫХ ДИССИПАТИВНЫХ СИСТЕМ}}
\author{Д.С. Загоруля \footnote{Vk, telegram: @dimicorn}}
\date{Весна 2022}
\usepackage{hyperref}
\hypersetup{colorlinks, citecolor=blue, filecolor=blue, linkcolor=blue, urlcolor=blue}
\usepackage{titlesec}
\titleformat{\section}{\filcenter\normalfont\MakeUppercase}{\thesection.}{.5em}{}
\titleformat{\subsection}[runin]{\normalfont\normalsize\bfseries}{}{0em}{}
%\renewcommand{\thesubsection}{}
\addto\captionsrussian{\renewcommand{\contentsname}{}}
\newtheorem{problem}{Задача}[section]
\theoremstyle{definition}
\newtheorem*{definition}{Определение}
%\newtheorem{definition}{Определение}[section]



\begin{document}
\maketitle
\tableofcontents
\newpage
\section{Программа курса}
\begin{enumerate}
    \item \textcolor{red}{Основные понятия.} Открытые системы. Неравновесность. \textcolor{red}{Нелинейный осциллятор,} возбудимая среда, автоволны, диссипативные структуры. Элементы качественной теории динамических систем. Диссипативные системы. \textcolor{red}{Фазовый портрет, траектория,} основные типы бифуркаций на плоскости. Устойчивость, характеристические показатели Ляпунова. Понятие аттрактора.
    \item Бифуркации в многомерных системах. \textcolor{red}{Странные аттракторы. Отображение Пуанкаре.} Теория одномерных гладких отображений. Хаос в динамических системах и сценарии (пути) его возникновения. \textcolor{red}{Фрактальные структуры и их размерность.}
    \item Примеры пространственно-временных структур в нелинейных распределенных системах. Автоволновые режимы в средах с диффузией. Уравнение Фишера-Колмогорова-Петровского-Пескунова. Волны перемещения. Бегущие импульсы. Спиральные волны. Ведущие центры. Стационарные неоднородные структуры, бифуркации Тьюринга.
    \item Пространственно-временные структуры в физических и химических системах и соответствующие модели. Модель Пригожина-Лефевра-Николиса (<<брюсселятор>>). Тепловые волны и неоднородные стационарные состояния в системе Fe$^{+}$H$_{2}$. Механизм эффекта баретирования. 
\end{enumerate}
\newpage
\section{Теоретические методы исследования диссипативных динамических систем}
\subsection{Динамическая система и ее состояние.}Под \textit{динамической системой} будем понимать объект или процесс, для которых однозначно определено понятие состояния как совокупности значений некоторых величин в заданный момент времени и задан оператор, определяющий эволюцию начального состояния во времени.

Например, система материальных точек с заданным потенциалом взаимодействия является типичным примером динамической системы, так как ее состояние полностью определяется значением начальных координат и импульсов всех точек, а эволюцию системы определяют классические уравнения движения (второй закон Ньютона). Более сложным примером является среда (в частности, атмосфера Земли, содержимое химического реактора и др.) в которой происходят процессы тепло- и массопереноса, физические фазовые переходы, химические реакции и т.п. Состояние такой в фиксированный момент времени определяется концентрацией фаз, температурой и другими параметрами, задаваемыми в каждой точке среды.
\subsection{Нелинейный осциллятор.} Рассмотрим динамическую систему, описываемую нелинейным эволюционным уравнением
\begin{equation*}
    \Ddot{x} + \sin{x} = 0.
\end{equation*}

Движение маятника можно рассматривать как движение в поле потенциальных сил с потенциалом $U(x) = - \cos{x}$, его график изображен на ... Там же изображены и соответствующие \textit{фазовые траектории} для различных начальных условий --- \textit{фазовый портрет} системы. Состояния равновесия нелинейного маятника на фазовой плоскости расположены на оси $OX$ в точках $x_{k} = \pi k, \ k = 0, \pm 1, \pm 2, ...$, при этом $x = 0, \pm 2\pi, \pm 4 \pi, ...$ --- точки устойчивого, а $x = \pm \pi, \pm 3 \pi, ...$ --- неустойчивого равновесия. Вблизи точек устойчивого равновесия фазовые кривые представляют собой замкнутые кривые, близкие к окружностям, что выражает тот факт, что при этом маятник совершает колебания, близкие к гармоническим. Через точки неустойчивого равновесия проходит кривая, называемая \textit{сепаратрисой}; она делит фазовую плоскость на области с качественно различным поведением. При росте энергии системы колебания переходят во вращение.\\
\textbf{Осциллятор Ван дер Поля.} Осциллятор Ван дер Поля описывается дифференциальным уравнением
\begin{equation*}
    \Ddot{x} - a(1-bx^{2})\dot{x} + x = 0, \ 0 < \tau \leq T,
\end{equation*}
здесь $a$ --- параметр возбуждения. Аналитически получить его решение не удается. На качественном уровне ясно, что при малых по модулю $x$ коэффициент при $\dot{x}$ отрицателен --- в системе возникает ``отрицательное трение'', отбрасывающее точку от начала координат. При больших же $|x|$ трение становится положительным, препятствующим росту координаты. Таким образом, в системе существует устойчивое состояние автоколебаний, к которому стремятся все фазовые траектории. Численное интегрирование уравнения движения осциллятора Ван дер Поля с различными начальными условиями дает фазовые траектории.
\subsection{Странные аттракторы.} При значении размерности фазового пространства $N \geq 3$ возможен принципиально новый тип фазовых траекторий --- это так называемые странные аттракторы. Фазовая траектория динамической системы в этом случае предствляет собой бесконечную линию без самопересечений, причем при $t \to \infty$ траектория не покидает заданной области и не притягивается ни к точкам равновесия, ни к циклическим траекториям. Примером такой системы является аттрактор Лоренца. Странному аттрактору соответствует сложное апериодическое движение, схожее с обыденным представлением о хаотическом процессе. Однако теоретически оно полностью предсказуемо и воспроизводимо --- для начальной задачи, определяющей странный аттрактор, могут выполняться условия теоремы существования и единственности, и, задав те же начальные условия и правые части дифференциальных эволюционных уравнений, мы воспроизведем то же самое решение, и значит --- ту же самую ``хаотическую'' траекторию. О таком поведении динамических систем говорят как о детерминированном динамическом хаосе.
\subsection{Разностные эволюционные уравнения.} В некоторых ситуациях для моделирования динамики системы достаточно указать ее состояние в заданные дискретные моменты времени. В этом случае в качестве эволюционного оператора можно использовать функцию, выражающую состояние системы в некоторый момент времени через ее состояние в предыдущий момент. Моделью такой динамической системы может служить следующее уравнение с заданным начальным условием:
\begin{equation}\label{eq: evol}
    \begin{split}
        \textbf{x}_{n+1} = \textbf{F}(\textbf{x}_{n});\\
        \textbf{x}_{0} = \textbf{x}_{*}.
    \end{split}
\end{equation}
Эволюционное уравнение в \eqref{eq: evol} можно получить из дифференциального уравнения, приведенного в ... заменой производной на разделенную разность: если, в соответствии с ..., $\dot{\textbf{x}} = \textbf{F}(\textbf{x})$, то переход к разностному уравнению дает соотношение $\textbf{x}_{n + 1} = \textbf{x}_{n} + \textbf{F}(\textbf{x}) \Delta t$, где $\Delta t$ --- шаг по времени, а $\textbf{x}_{n}$ --- значение параметров системы $\textbf{x}(t)$ в момент времени $t = t_{n}, \ n = 1, 2,...$

Если фазовое пространство одномерно, то график правой части эволюционного уравнения \eqref{eq: evol} --- функции $F(x)$ --- дает наглядное представление об эволюции системы. В частности, если это график пересекает диагональ первой и третьей координатных четвертей, то координаты этой точки пересечения дают стационарную точку динамической системы, так как условием пересечения графика с диагональю первого и третьего координатных углов является соотношение $x_{n+1} = x_{n}$. Динамику систему удобно проследить графически.
\subsection{Отображение Пуанкаре.} Иногда для исследования системы анализ ее состояний во все моменты времени $t > 0$ является, с одной стороны, затруднительным, а с другой стороны, явно избыточным. ТОгда удобно дискретное описание динамической системы, возникающее естественным образом в следующей ситуации. Рассмотрим в фазовом пространстве $\mathcal{R}_{N}$ поверхность $\mathcal{L}_{N-1}$ размерности $\dim{\mathcal{L}_{N-1}} = N - 1$, на единицу меньшей, чем размерность $\dim{\mathcal{R}_{N}} = N$ фазового пространства $\mathcal{R}_{N}$, пусть уравнение этой поверхности задано в виде $S(\textbf{x}) = 0$ и фазовая траектория $x(t), \ t \geq 0,$ пересекает ее под углом, отличным от нуля (т.е. не по касательной, или, как говорят, трансвервально), так, что при пересечении знак $S(\textbf{x}(t))$ изменяется, например, с плюса на минус. Обозначим время такого первого пересечения $t_{1}$, следующего --- $t_{2}$ и т.д. Уравнения $\textbf{x}(t_{k}) = P(\textbf{x}(t_{k-1}))$, связывающее координаты точек двух последующих пересечений, называется отображением Пуанкаре, а поверхность $\mathcal{L}_{N-1}$ --- сечением Пуанкаре.

С помощью отображения Пуанкаре описание динамики становится проще, так как уменьшается размерность модели (точки на сечении Пуанкаре имеют размерность $N-1$), и вместо непрерывного времени рассматривается дискретный набор $t_{1}, t_{2}, ...$ его значений. Само же описание в терминах отображения Пуанкаре в ряде случаев очень удобно. Например, стационарная точка отображения Пуанкаре, $\textbf{x}_{k+1} = P(\textbf{x}_{k})$, соответствует циклическому движению системы по замкнутой траектории.
\subsection{Примеры бифуркаций.} \textbf{Смена устойчивости.}
Рассмотрим динамическую систему с фазовым пространством размерности единица, эволюция которой описывается уравнением
\begin{equation}\label{eq: xdot}
    \dot{x} = F(x, \mu),
\end{equation}
и пусть $x_{0}$ --- состояния равновесия, т.е.
\begin{equation}\label{eq: F}
    F(x_{0}, \mu) = 0.
\end{equation}

Тогда, если $F'_{x}(x_{0}, \mu) \not = 0$ и функция $s(\mu) = F'_{x}(x_{0}, \mu)$ непрерывна по $\mu$, так, что уравнение \eqref{eq: F} разрешимо относительно $x_{0}$ в окрестности точки $\mu$, то состояние равновесия $x_{0}$ является грубым, так как малым изменением параметра $\mu$ мы не изменим в качественном отношении фазовый портрет системы --- \textit{точка равновесия не исчезает и новых точек равновесия не появляется}. Линеаризованное уравнение для \eqref{eq: xdot} записывается в виде
\begin{equation*}
    \dfrac{\partial}{\partial t}(x-x_{0}) = F'_{x}(x_{0}, \mu)(x-x_{0}),
\end{equation*}
и устойчивость точки $x_{0}$ определится знаком первой производной: при $F'_{x}(x_{0}, \mu) < 0$ равновесие устойчиво, а при $F'_{x}(x_{0}, \mu) > 0$ --- неустойчиво. Если же при некотором значении $\mu_{0}$ параметра $\mu$ производная $F'_{x}(x_{0}, \mu)$ обращается в нуль: $F'_{x}(x_{0}, \mu) = 0$, то при переходе через точку $\mu = \mu_{0}$ имеет место бифуркация смены устойчивости.

Например, пусть динамическая система описывается эволюционным уравнением
\begin{equation*}
    \dot{x} = \mu \sin{x}.
\end{equation*}
Тогда при $\mu > 0$ точка $x = 0$ задает грубое устойчивое состояние равновесия, при $\mu < 0$ точка $x = 0$ --- грубое неустойчивое состояние равновесия, а при $\mu = 0$ происходит бифуркация смены устойчивости.

Заметим, что значение производной $F'_{x}(x_{0}, \mu)$ является характеристическим ляпуновским показателем динамической системы.\\
\textbf{``Седло-узел''. Складка.} Пусть в эволюционном уравнении \eqref{eq: xdot} правая часть задана равенством
\begin{equation*}
    F = -\mu_{1} + \mu_{2}x^{2},
\end{equation*}
и для определенности $\mu_{2} > 0$. Тогда при $\mu_{1} > 0$ в системе существуют два положения равновесия --- устойчивое и неустойчивое; при $\mu_{1} = 0$ они сливаются в одно, а при $\mu_{1} > 0$ --- исчезают. В комбинированном пространстве фазовой координаты и параметров $\{\mu_{1}, \mu_{2}\}$ бифуркацонная диаграмма выглядит как складка; здесь координата точки поверхности, отложенная по вертикальной оси, задает положение точки равновесия системы. Бифуркация $\mu_{1} = 0$ называется \textit{бифуркацией срыва равновесия или седло-узловой бифуркацией} (в точке $\mu_{1} = 0$ устойчивый и неустойчивый узлы сливаются в одну точку равновесия системы и исчезают при дальнейшем изменении параметра $\mu_{1}$. Ситуация, когда по одному направлению возмущения система является устойчивой, а по другому --- неустойчивой, называется \textit{седловой}). Она также называется \textit{бифуркацией коразмерности единица}, так как выделяется единственным условием: в точки бифуркации $F'_{x}(x_{0}, \mu_{0}) = 0$.\\
\textbf{Сборка.} Рассмотрим уравнение $\dot{x} = c + \mu_{1}x + \mu_{2} x^{3}$, задающее эволюцию системы, где по-прежнему, как и в предыдущем разделе, $\mu_{2} > 0$. В зависимости от значения параметра $\mu_{1}$ в системе может существовать либо три, либо одно грубое состояние равновесия.

На рис. ... такая ситуация изображается поверхностью типа сборки. Изменение параметров $\mu_{1}$ и $\mu_{2}$ вдоль кривых ($\mu_{1},\mu_{2}$), идущих под ненулевым углом к границе заштрихованной области --- бифуркационным линиям $l_{0}$ (как говорят, \textit{трансверсально} к $l_{0}$), характеризуется гистерезисом: при движении справа налево система скачком меняет свое устойчивое равновесие при значениях параметров, соответствующих точке $A_{1}$ на плоскости параметров $\{\mu_{1}, \mu_{2}\}$, а при возвращении по той же линии скачкообразное изменение устойчивого равновесия происходит в точке $A_{2}$.

В точке $B$ выполняются два равенства $F'_{x} = 0$, $F''_{xx} = 0$, поэтому говорят, что при значениях параметров, соответствующих точке $B$, происходит \textit{бифуркация коразмерности два}.

Складка и сборка являются элементарными особенностями поверхностей, из которых может быть скомбинирована любая особенность поверхностей в трехмерном пространстве $(x, \mu_{1}, \mu_{2})$.\\
\textbf{Рождение предельного цикла.}\\
\textbf{Удвоение периода и расщепление цикла}\\
\subsection{Примеры и определения фрактала.}
Строго определить фрактал как математический объект не удается, есть лишь попытки дать такое определение. Наиболее известными являются определения Бенуа Мандельброта, математика, благодаря работам которого теперь осознается, насколько важны эти новые геометрические объекты для понимания окружающего мира. В основе первого, пробного определения лежит представление о топологической размерности множеств: размерность точки принимается равной нулю, линии --- единицу, плоской фигуры --- двум и т.п. Формулируется оно так: ``\textit{Фракталом} называется такое множество, размерность Хаусдорфа-Безиковича которого строго больше его топологической размерности''. ``Дробность размерности'' и выражает ``пограничное'' свойство фракталов лежать между точкой и линией, или между линией и поверхностью и т.д. Однако, мало того, что требуется расшифровка понятия дробной размерности Хаусдорфа-Безиковича, неудачность его стала очевидной после приведения ряда контрпримеров объектов, для которых это определение не выполняется, хотя, исходя из интуитивного представления, имело бы смысл отнести к фракталам (например, чрезвычайно ``дырявая'' пирамида, построенная польским математиком, топологом Вацлавом Серпинским, формально имеет размерность, равную двум, хотя получена из трехмерного тетраэра поочередным отбрасыванием вписанных в него тетраэдров с половинной стороной).

Несколько менее формальное и значительно более общее определение фрактала, данное Б. Мандельбротом несколько позже звучит так: ``\textit{Фракталом} называется структура, состоящая из частей, которые в некотором смысле подобны целому''. Неопределенность этого определения, содержащаяся в словах ``в некотором смысле'', делает понятие фрактала чуть ли не всеобъемлющим.

Поясним, как в это определение укладываются ``математические'' фракталы типа прямой фон Коха \footnote{Прямая Коха получается из отрезка прямой последовательной заменой каждого прямолинейного участка на ломаную путем ``вытягивания'' средней трети исходного отрезка до равностороннего треугольника. Повторяя такую процедуру бесконечного число раз, в пределе получим конечную ``линию'', соединяющую две точки, но имеющую бесконечную длину.}. Для этого вначале заметим, что такие геометрические объекты, как прямая или плоскость, разумно назвать самоподобными. Формально охарактеризуем это свойство можно тем, что эти фигуры не изменяютсяя при некоторых геометрических преобразованиях: перенос прямой вдоль нее приводит к той же самой прямой, плоскость переходит в себя при параллельном сдвиге и повороте. Независимость от преобразований в математике называется симметрией. Есть множества, не обладающими столь полной симметрией, как плоскость или прямой, например, окружность не изменяется только при повороте --- она тоже самоподобна. В этом смысле, согласно второму определению, все эти множества являются фракталами, несмотря на свою простую геометрическую структуру. Их можно назвать ``гладкими фракталами'', в отличие от кривой Коха, пирамиды Серпинского, множества Кантора и т.п. Какой же симметрией обладает кривая Коха? Выбрав ее фрагмент, например, одну треть всей кривой, и увеличив его в три раза, мы вновь получим в точности исходную кривую.
\subsection{Динамические фракталы.}
Фрактальность поведения сложных нелинейных систем также может рассматриваться как их неотъемлемое свойство. Если система достаточно сложна, то она в своем развитии обязательно проходит через чередующиеся этапы устойчивого и хаотического развития. Причем сценарии перехода от порядка к хаосу и обратно поддаются классификации, и вновь все многообразие природных процессов распадается на небольшое число качественно подобных. Сначала, при некотором значении коэффициента пропорциональности, в системе имеется лишь одно устойчивое положение равновесия. При изменении коэффициента $r$ наступает момент, когда точка равновесия раздваивается, возникает два устойчивых состояния, в которых система пребывает по очереди, то в одном, то в другом, шаг за шагом по времени. Потом каждая из этих вновь раздваивается, и ситуация повторяется, сохраняя общий рисунок. Рано или поздно множество точек равновесия плотно заполняет все множество состояний, система переходит к хаосу, полностью разрушая свою структуру. Но затем, при дальнейшем росте параметра, из хаоса вновь возникает некоторое конечное число упорядоченных состояний, которые в конце концов ``схлопываются'' в единственное, и все начинается сначала. Напомним, что в математической модели этого явления обнаружено множество подобных элементов; эти свойства подобия носят универсальности Фейгенбаума.
\subsection{Размерность Хаусдорфа-Безиковича.}
Для определения размерности Хаусдорфа-Безиковича рассмотрим ряд примеров. Зададимся вопросом: как измерить величину множества точек $\mathcal{L}$ метрического пространства? Разобьем пространство на ячейки с характерным размером (диаметром) $\delta$ (это могут быть шары, кубы и др.) и подсчитаем число ячеек, покрывающих множество. Уменьшая размер ячеек и следя за скоростью возрастания их числа, необходимого для покрытия множества $\mathcal{L}$, можно получить представление о размерности $\mathcal{L}$, в частности, вычислить ``длину'' множества, его ``площадь'', ``объем'' и т.п. Действительно, пусть $\mathcal{L}$ --- спрямляемая кривая длины $L_{0}$. Выберем минимальное покрытие, т.е. такое, которое состоит из наименьшего числа ячеек (существование такого покрытия для компактного множества следует из леммы Гейне-Бореля). Число ячеек $N(\delta)$ в этом покрытии будет пропорционально отношению $L_{0}/\delta$, и длину кривой получим предельным переходом при $\delta \rightarrow 0$:
\begin{equation*}
    L = N(\delta)\delta \sim \dfrac{L_{0}}{\delta}\delta \xrightarrow[\delta \rightarrow 0]{} L_{0}.
\end{equation*}
Множеству точек, представляющих собой кривую длины $L_{0}$, можно сопоставить и площадь. Так как каждая ячейка имеет характерную площадь $\delta^2$, то площадь множества пропорциональна $\delta^2 N(\delta)$, и при $\delta \rightarrow 0$ получим, что
\begin{equation*}
    S = N(\delta) \delta^2 \sim \dfrac{L_{0}}{\delta}\delta^2 = L_{0} \delta \xrightarrow[\delta \rightarrow 0]{}0.
\end{equation*}
Аналогично можно вычислить объем множества точек и т.д.

Если в качестве множества $\mathcal{L}$ взять часть квадрируемой поверхности площади $S$, то число ячеек, необходимое для покрытия этой поверхности, пропорционально $N(\delta) = S/\delta^2$; ``длина'' поверхности определяемая предельным переходом, равна
\begin{equation*}
    L = N(\delta) \delta \sim \dfrac{S}{\delta^2}\delta = \dfrac{S}{\delta} \xrightarrow[\delta \rightarrow 0]{} \infty,
\end{equation*}
площадь $S$, $0 < S < \infty$, дается соотношением
\begin{equation*}
    S = N(\delta) \delta^2 \sim \dfrac{S}{\delta^2}\delta^2 = S,
\end{equation*}
а объем равен нулю:
\begin{equation*}
    V = N(\delta) \delta^3 \sim \dfrac{S}{\delta^2}\delta^3 = S \delta \xrightarrow[\delta \rightarrow 0]{}0.
\end{equation*}

В этих примерах для характеристики ``величины'' множества точек $\mathcal{L}$ используется некоторая пробная функция $h(\delta) \sim \gamma(d) \delta^{d}$, которая определяет величину ячейки, --- длину при $d = 1$, площадь при $d = 2$, объем при $d = 3$ и т.п., и ``величина'', или мера, множества $\mathcal{L}$ определяется как сумма ``величин'' всех ячеек, покрывающих $\mathcal{L}$:
\begin{equation*}
    M_{d} = \sum h(\delta).
\end{equation*}
Здесь константа $\gamma(d)$ зависит от формы ячейки и от того, какое свойство множества характеризует функция $h(\delta)$; например, если $h(\delta)$ --- площадь круга, то $\gamma(d) = 2 \pi$, а если площадь квадрата, то $\gamma(d) = 1$. При некотором $d$ соответствующая мера $M_{d}$ при $\delta \rightarrow 0$ равна либо 0, либо бесконечности, либо некоторому конечному положительному числу, и то значение $d$, при котором $M_{d}$ не равна ни нулю, ни бесконечности, адекватно отражает топологическую размерность множества $\mathcal{L}$. Заметим, что результат не изменится, если в качестве значений $d$ выбирать не только целые числа, но и любые действительные. Это соображение позволяет обобщить понятие размерности, и дать следующее определение размерности, называемой размерностью Хаусдорфа-Безиковича:
\begin{definition}
    Число $d_{cr}$, такое, что
\end{definition}
\begin{equation*}
    \lim \limits_{\delta \to 0} M_{d} = 
    \begin{cases}
        0, & d > d_{cr}\\
        \infty, & d < d_{cr}
    \end{cases},
\end{equation*}
называется \textit{размерностью Хаусдорфа-Безиковича}.

Здесь пока расмотрены лишь достаточно простые множества, для которых размерность Хаусдорфа-Безиковича совпадает с топологической размерностью, однако существуют и более сложные геометрические объекты, для которых скачок меры $M_{d}$ от нуля до бесконечноти происходит при нецелых значениях $d$. К таким объектам относится, в частности, и расмотренная ранее кривая Коха.

Более наглядное представление о размерности Хаусдорфа-Безиковича можно получить, рассматривая $N(\delta)$ как функцию $\delta$ со степенной особенностью в нуле:
\begin{equation*}
    N(\delta) = \dfrac{\text{const}}{\delta^{d}} + \alpha(\delta),
\end{equation*}
где $\alpha(\delta) \delta^{d} \rightarrow 0$ при $\delta \rightarrow 0$. Размерность Хаусдорфа-Безиковича, как легко видеть, равна степени особенности $d$, в этом смысле размерность Хаусдорфа-Безиковича определяет скорость роста числа элементов минимального покрытия множества $\mathcal{L}$ при стремлении размера элемента к нулю. Из предыдущей формулы получим, что при критическом значении $d$ с точностью до бесконечно малых величин выполняется соотношение $N(\delta) \delta^{d} = \text{const}$, откуда
\begin{equation*}
    d_{cr} = \lim \limits_{\delta \to 0} \dfrac{\ln{N(\delta)}}{\ln{\frac{1}{\delta}}}.
\end{equation*}

Размерность Хаусдорфа обобщается в приведенном здесь определении. Обобщение касается того, что размерность Хаусдорфа-Безиковича не зависит от формы ячейки --- будь то сфера, эллипсоид, куб или ячейка другой формы.\\
\subsection{Примеры размерностей фракталов.}\textbf{Размерность кривой Коха.} Вообще говоря, определение фрактала для этой кривой будет выполняться лишь в пределе, когда длина отрезка прямой стремится к нулю, результат же, полученный на $n$-ом шаге, называетя предфракталом; предфрактал представляет собой обычную спрямляемую линию.

При $\delta = 1/3$ число элементов покрытия равно $N(\delta) = 4$, а при $\delta = (1/3)^{n}$ --- соответственно $N(\delta) = 4^{n}$, поэтому размерность Хаусдорфа-Безиковича кривой фон Коха равна
\begin{equation*}
    D = \frac{\ln{4}}{\ln{3}} > 1.
\end{equation*}
\textbf{Размерность салфетки Серпинского.} Вацлав Серпинский предложил пример ``плоского'' фрактального множества, получаемого из правильного треугольника последовательными выбрасыванием средних частей.

Размерность салфетки Серпинского легко подсчитать по формуле, выбирая в качестве элемента покрытия правильный треугольник со стороной $\delta = (1/2)^{n}$, тогда $N(\delta) = 3^{n}$, а значит, $D = \ln{3}/\ln{2}$. Заметим, что в этом случае $1 < D < 2$, и если считать ``топологическую размерность'' салфетки Серпинского равной двум (так как она получена из плоской фигуры), то первое определение фрактала не выполнено. С другой стороны, так как суммарная площадь всех выкинутых треугольников равна общей площади исходного треугольника, такое множество трудно назвать плоским, т.е. имеющим топологическую размерность, равную двум. Все это демонстрирует трудности, возникающие на пути применения первого определения фрактала.\\
\textbf{Ковер Серпинского.} Это тоже плоская фрактальная фигура, полученная способом, аналогичным способу получения салфетки Серпинского, но начальным элементом здесь является единичным квадрат. При построении ковра Серпинского на первом шаге единичный квадрат делится на девять маленьких квадратов длиной стороны, равной 1/3, выбрасывается центральный квадрат и процедура повторяется с оставшимися квадратами бесконечное число раз. Размерность ковра Серпинского равна $D = \ln{8}/\ln{3}$, и $1 < D < 2$.\\
\textbf{Трехмерный аналог салфетки Серпинского.} Определение фрактала с дробной размерностью Хаусдорфа-Безиковича тоже не выдерживает критики, что демонстрируется примером пирамиды Серпинского. Это --- объемная фрактальная фигура, полученная способом, аналогичным способу получения плоской салфетки Серпинского, но начальным элементом здесь является правильный тетраэдр с единичной длиной ребра. Фигура получается из тетраэдров, последовательно отсекаемых от вершин исходного тетраэдра, стороны отсекаемых тетраэдров равны половине стороны тех тетраэдров, от которых они отбрасываются. При элементе покрытия в виде тетраэдра с длиной стороны $\delta = (1/2)^{n}$ требуется $N(\delta) = 4^{n}$ элементов покрытия, что приводит к размерности пирамидки Серпинского, равной двум. Итак, размерность этого объекта --- целое число, хотя назвать его ``плоским'' вряд ли возможно.



\newpage
\section{Контрольные вопросы и задачи}
\begin{enumerate}
    \item Провести бифуркационный анализ модели <<брюсселятор>>.
    \item Доказать, что сумма показателей Ляпунова равна усредненной по траектории дифергенции векторного поля.
    \item Построить канторово множество, имеющее фрактальную размерность $\log_{7} 4$.
    \item Чему равна фрактальная размерность трехмерного ковра Серпинского?
    \item Вычислить два первых бифуркационных параметра последовательности Фейгенбаума для логистического отображения.
    \item Найти скорость установившейся волны (автомодельного решения) для уравнения $\frac{\partial x}{\partial t} = \alpha (a-x)(b-x)(c-x) + \frac{\partial^2 x}{\partial r^2}$, где $a < b < c$.
    \item Вывести условия бифуркации Тьюринга для распределенной модели <<брюсселятор>>.
    \item В чем суть эффекта баретирования? Оценить диапазон напряжения, в котором ток, протекающий через баретор, остается постоянным.
\end{enumerate}
\section{Список литературы}
\begin{enumerate}
    \item А.Ю. Лоскутов, А.С. Михайлов. Введение в синергетику. М.: Наука, 1990.
    \item А.И. Чуличков. Математические методы нелинейной динамики. М.: Физматлит, 2000.
    \item Марри Дж. Нелинейные дифференциальные уравнения в биологии. Лекции о моделях. Пер. с англ. М.: Мир, 1983.
    \item А. Лихтенберг, М. Либерман. Регулярная и стохастическая динамика. М.: Мир, 1984.
    \item В.А. Васильев, Ю.М. Романовский, В.Г. Яхно. Автоволновые процессы. М.: Наука, 1987.
\end{enumerate}


\end{document}

